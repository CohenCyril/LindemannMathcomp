\documentclass[11pt,a4paper,draft]{article}
\usepackage[utf8]{inputenc}
\usepackage[T1]{fontenc} 
\usepackage[english]{babel}   
\usepackage{booktabs}
\usepackage{amssymb,amsmath,amsthm,amsfonts}
\usepackage{float}
\usepackage{graphicx}
\usepackage{xspace}
\usepackage[hyphens]{url}
\usepackage[hidelinks]{hyperref}
\hypersetup{breaklinks=true}
\urlstyle{same}
%\usepackage{stmaryrd}
\usepackage{mathrsfs}
\usepackage{enumerate}
%\usepackage{wrapfig}
\usepackage{cite}
%\usepackage{sidecap}
\usepackage{caption}
%\usepackage{makecell}
\usepackage[svgnames]{xcolor}
\usepackage{colortbl}
\usepackage{multicol,multirow}
\usepackage{textcomp}
\usepackage{pdfpages}
\usepackage[hmarginratio=3:2,marginparwidth=0pt]{geometry}
%\usepackage{changepage}
\usepackage[lighttt]{lmodern}
\usepackage{listings}
%\usepackage{adjustbox}
%\usepackage{bold-extra}
\usepackage{fancyhdr}
\usepackage{textcomp}
%\usepackage{emptypage}
%\usepackage{layout}
%\renewcommand\theadalign{cb}
%\renewcommand\theadfont{\bfseries}
%\renewcommand\theadgape{\Gape[4pt]}
%\renewcommand\cellgape{\Gape[4pt]}

\xdefinecolor{abricot}{named}{DarkGreen}

\newcommand{\textapprox}{\raisebox{0.5ex}{\texttildelow}}
\newcommand{\HRule}{\rule{\linewidth}{0.5mm}}
\setlength{\parindent}{10pt}

\renewcommand{\thesection}{\arabic{section}}

\makeatletter
\newcommand*{\toccontents}{\@starttoc{toc}}
\makeatother

%\newcounter{code}[chapter]

\newenvironment{scode}{
\footnotesize%
\ttfamily%
\begin{adjustwidth}{2.2em}{0em}%
}
{%
\end{adjustwidth}%
\normalfont%
\normalsize%
}



\newenvironment{tcode}[2][width=\linewidth]%
{%
\begin{adjustbox}{#1}
\footnotesize%
\ttfamily%
\begin{tabular}{#2}%
}%
{%
\end{tabular}%
\normalfont%
\normalsize%
\end{adjustbox}%
}




\newcommand{\codecaption}[1]{
\refstepcounter{code}
\addtocounter{table}{-1}
\begin{center}Code \thechapter .\arabic{code}: #1\end{center} }

\newcommand{\codt}{\hspace{2em}}

\newcommand{\textstt}[1]{%
\normalsize\footnotesize\texttt{#1}\normalsize}

%\usepackage{tocloft}

\renewcommand{\contentsname}{Contents}



%\renewcommand{\cfttoctitlefont}{\Huge\bfseries}
%\renewcommand{\cftaftertoctitle}{\hfill}
%\setlength{\cftbeforetoctitleskip}{0pt}
%\setlength{\cftaftertoctitleskip}{20pt}









\newcounter{ctheorem}[section]
\newcounter{clemma}[section]
\newcounter{cprop}[section]



\newtheorem{thm}{Theoreme}[ctheorem]
\newtheorem{lem}[clemma]{Lemme}
\newtheorem{prop}[cprop]{Proposition}













\begin{document}

\section{Notations}

\paragraph{Polynômes}
Soit $P$ un polynôme dans un anneau $\mathbb{K}$ quelconque. On note alors $\text{deg} P$ son degré, et ses coefficients sont alors $p_i$ pour $i$ compris entre $0$ et $\text{deg} P$. On a notamment : $P = \sum\limits_{i = 0}^{\text{deg} P} p_i X^i$.

\paragraph{Nombres algébriques}
Un nombre $x \in \mathbb{C}$ est dit algébriques s'il est racine d'un polynôme à coefficients entiers. On note l'ensemble de ces nombres $\mathbb{A}$ :
$$\mathbb{A} = \{ x \in \mathbb{C} \mid \exists P \in \mathbb{Z}[X], P(x) = 0\}.$$
De façon complètement équivalente, on peut se contenter de demander au polynôme d'être à coefficients dans $\mathbb{Q}$ :

\begin{prop}
$\mathbb{A} = \{ x \in \mathbb{C} \mid \exists P \in \mathbb{Q}[X], P(x) = 0\}.$
\end{prop}

\begin{proof}[Preuve. ]
Par double inclusion, on note $B = \{ x \in \mathbb{C} \mid \exists P \in \mathbb{Q}[X], P(x) = 0\}$.
\begin{description}
\item[$\mathbb{A} \subset B$ :] Soit $x \in A$, et $P \in \mathbb{Z}[X]$ tel que $P(x) = 0$. En particulier, $P \in \mathbb{Q}[X]$.
\item[$B \subset \mathbb{A}$ :] De même, pour $x \in B$, il existe un polynôme $P \in \mathbb{Z}[X]$ vérifiant $P(x) = 0$. On pose alors $c = \prod_{i = 0}^{\text{deg} P} \text{den}(P_i)$, le produit des dénominateurs des coefficients de P. Finalement, $cP$ est maintenant à coefficients dans $\mathbb{Z}[X]$, et on a toujours $cP (x) = 0$, donc $x \in \mathbb{A}$.
\end{description}
\end{proof}

\paragraph{Fonctions injectives}
Pour $n$ et $p$ des entiers naturels non nuls, vérifiant $n \geq p$, on note $\mathfrak{S}_{n,p}$ l'ensemble des fonctions injectives de $[1; n]$ dans $[1; p]$.

\paragraph{Permutations}
Pour $n$ un entier naturel, $\mathfrak{S}_{n,n}$ représente les fonctions injectives, et donc bijectives de $[1; n]$ dans lui-même, c'est à dire l'ensemble des permutations de $[1; n]$. On le note plus simplement $\mathfrak{S}_n$.

\section{Théorème de Lindemann}

\begin{thm}[Lindemann]
Soient $l \in \mathbb{N}^*$, $a_1, \ldots a_l$ des nombres algébriques, tous non nuls, et $\alpha_1, \ldots \alpha_l$ des nombres algébriques deux à deux distincts. Alors :
$$ \sum\limits_{i = 1}^l a_i e^{\alpha_i} \neq 0.$$
\end{thm}

Dans la suite, on privilégie les lettres grecques pour dénoter un ensemble de nombres algébriques deux à deux distincs.

La preuve se constitue de différentes étapes, dont les premières consistent à ajouter des hypothèses sans perte de généralités.

\subsection{Première simplification : les $a_i$}

\begin{lem}
Soient $m \in \mathbb{N}^*$, $c_1, \ldots c_m$ des \emph{nombres entiers}, tous non nuls, et $\alpha_1, \ldots \alpha_n$ des nombres algébriques deux à deux distincts. Alors :
$$ \sum\limits_{i = 1}^n a_i e^{\alpha_i} \neq 0.$$
\end{lem}

\begin{proof}[Preuve : Lemme 1 $\implies$ Theoreme 1.]
Par contraposée.

Soient $l \geq 1$, $(a_1, \ldots, a_l) \in \mathbb{A} \setminus \{0\} ^l$, et  $(\alpha_1, \ldots \alpha_l) \in \mathbb{A}^l$, tels que : $\forall i, j \in [ 1 ; l ]^2 \text{, } \alpha_i = \alpha_j \implies i =j \text{ et } %TODO llbracket/rrbracket
 \sum_{i = 1}^l a_i e^{\alpha_i} = 0.$
 
 
Pour tout $i$ compris entre $1$ et $l$, $a_i$ est algébrique. On note alors $P_i \in \mathbb{Z}[X]$ son polynôme minimal. En particulier, $P_i \neq 0$ puisque c'est un polynôme minimal.

On pose $P = \prod_{i = 1}^l P_i \in \mathbb{Z}[X]$ le produit de tous ces polynômes minimaux, et on note $L = \text{deg} P$ son degré. Les racines de $P$ comptées avec leur multiplicité sont appelées les $b_j$, pour $1 \leq j \leq L$, et sont donc au nombre de $L$, puisque $\mathbb{C}$ est un corps algébriquement clos, et elles sont algébriques.

On va alors étudier $\prod_{\sigma 







\end{proof}



\end{document}




